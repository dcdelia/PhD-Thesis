\section{Interprocedural Profiling}

The first contribution we present in this thesis is an {\em interprocedural} technique for identifying the calling context that are most frequently encountered across function invocations at run-time. We show that the traditional approach for constructing a {\em Calling Context Tree} (CCT) might not be sustainable for real-world applications, as their CCTs often consist of tens of millions of nodes, making them difficult to analyze and also hurting execution time because of poor access locality. We thus introduce a novel data structure, the {\em Hot Calling Context Tree} (HCCT), in the spectrum of representations for interprocedural control flow. The HCCT is defined as the subtree of the CCT containing only its most visited nodes, which we call {\em hot}, along with their ancestors, and can be constructed independently of the CCT using fast, space-efficient algorithms for mining frequent items in data stream.

\subsection{Motivation and Contributions}
The dynamic {\em calling context} of a routine invocation is defined as the sequence of functions that are concurrently active on the run-time stack. A calling context leads to an exact program location, as it corresponds to the sequence of un-returned calls from a program’s root function to the routine invocation it is associated with.

Context-sensitive profiling information provides valuable information for program understanding, performance analysis, and runtime optimizations. Previous works demonstrated its effectiveness for tasks such as residual testing~\cite{PavlopoulouY99,Vaswani07}, function inlining~\cite{Chang92}, statistical bug isolation~\cite{Feng03,Liblit03}, performance bug detection~\cite{Nistor13}, object allocation analysis~\cite{Nethercote07}, event logging~\cite{Zhang06}, or anomaly-based intrusion detection~\cite{Bond07}.
% this is a sync with PCCE paper
Calling-context information can also be employed in unit test generation~\cite{Villazon09}, testing of sensor network applications~\cite{Lai08}, and reverse engineering of protocol formats~\cite{Lin08}.

\mynote{Add reference to related work section for the CCT}

Calling context trees (CCTs) offer a compact representation for context-sensitive information, and many techniques have been proposed to reduce the overhead for CCT construction by trading accuracy for performance. However, as noticed in previous works, CCTs may be very large and difficult to analyze in several applications. Moreover, their sheer size might hurt execution time because of poor access locality during construction and query. Under the optimistic assumption that each CCT nodes requires 20 bytes for its representation on 32-bit architectures, numbers shown in {\bf XXX} already result in almost 1 GB needed just to store a CCT with 48 million nodes for an application such as OpenOffice Calc.

%~\cite{}

%Context-sensitive profiling provides 
%These algorithms allow us to distinguish between hot and cold context on-the-fly, and we show both theoretically and experimentally that for collected metrics the HCCT achieves a similar precision as the CCT in a space that is several orders of magnitude smaller. We show on prominent benchmarks that our implementation, shipping as a plugin for the \gcc\ compiler, incurs a slowdown competitive with the \gprof\ profiler while collecting much finer-grained profiles.

\subsection{Approach}

\subsection{Algorithms}

\subsection{Implementation}

\subsection{Comparison with Related Work}

\subsection{Discussion}
