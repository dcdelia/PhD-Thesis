\subsection{Language Framework}

In this section we introduce some basic definitions used in our representation of programs and transformations. In particular, we provide a syntax and a big-step semantics for a simple imperative language, and introduce means for reasoning about program properties using computation tree logic operators and for describing program transformations through rewrite rules with side conditions.

\subsubsection*{Program Syntax and Semantics}

Our discussion is based on a minimal imperative programming language with instructions and expressions with the syntax \ifauthorea{below}{in \myfigure\ref{fig:osr-program-syntax}}.

\ifdefined\noauthorea
\begin{figure}[t]
\fi
\noindent
\begin{small}
$
\begin{array}{rcl}
\texttt{$Instr$} & ::= & \hphantom{\texttt{| }}\texttt{$Var$ := $Expr$} \\
& & \texttt{| if ( $Expr$ ) goto $Num$} \\
& & \texttt{| goto $Num$} \\
& & \texttt{| skip} \\
& & \texttt{| abort} \\
& & \texttt{| in $Var\cdots Var$} \\ 
& & \texttt{| out $Var\cdots Var$} \\
\texttt{ $Expr$ } & ::= & \texttt{$Num$ | $Var$ | $Expr$ + $Expr$ | $\ldots$ } \\
\texttt{ $Var$ } & ::= & \texttt{X | Y | Z | $\ldots$} \\
\texttt{ $Num$ } & ::= & \texttt{$\ldots$ | -2 | -1 | 0 | 1 | 2 | $\ldots$} \\
\end{array}
$
\end{small}
\ifdefined\noauthorea
\caption{\label{fig:osr-program-syntax}Program Syntax}
\end{figure}
\fi

\begin{definition}[Program]
\label{de:program}
A program is a sequence of instructions the form:
\vspace{-3mm}
\begin{equation*}
\pi=\langle I_1, I_2, \ldots, I_n \rangle\in Prog = \bigcup_{i=2}^{\infty} Instr^{i}
\vspace{-4mm}
\end{equation*}
where: 

\begin{itemize}[itemsep=3pt,parsep=0pt,topsep=3pt]
%\item $Instr$ is defined by the syntax of Figure~\ref{fi:program-syntax}
\item \texttt{$I_i\in Instr$} is the $i$-th instruction of the program, indexed by program point \texttt{$i\in[1,n]$}
\item \texttt{$I_1$ $=$ in $\cdots$} is the initial instruction
\item \texttt{$\forall i\in[2,n-1]:$ $I_i$ $\neq$ in $\cdots$ $\wedge$ $I_i$ $\neq$ out $\cdots$}
\item \texttt{$I_n$ $=$ out $\cdots$} is the final instruction
\end{itemize}
\end{definition}

\noindent Instruction \texttt{in}, which must appear at the beginning of a program, specifies the variables that must be defined prior to entering the program. Similarly, \texttt{out} occurs at the end and specifies the variables that are returned as output. 

By \texttt{e[x]} we indicate that \texttt{x} is a variable of expression \texttt{e}\,$\in Expr$. We also denote by $vars($\texttt{e}$)$ the set of variables that occur in expression \texttt{e}. By $|\pi|=n$ we indicate the number of instructions in $\pi=\langle I_1, I_2, \ldots, I_n \rangle$.

\begin{definition}[Memory Store]
A {\em memory store} is a total function $\sigma:Var\rightarrow \mathbb{Z}\cup\{\bot\}$ that associates integer values to defined variables, and $\bot$ to undefined variables. We denote by $\Sigma$ the set of all possible memory stores. 
\end{definition}

\noindent By $\sigma[\wx\gets v]$ we denote the same function as $\sigma$, except that $\wx$ takes value $v$. Furthermore, for any $A\subseteq Var$, $\sigma\vert_{A}$ denotes $\sigma$ restricted to the variables in $A$, i.e., $\sigma\vert_{A}(\wx)=\sigma(\texttt{x})$ if $\wx\in A$ and $\sigma\vert_{A}(\wx)=\bot$ if $\wx\not\in A$. 

\begin{definition}[Program State]
\label{de:prog-state}
The {\em state} of a program $\pi=\langle I_1, I_2, \ldots, I_n \rangle$ is described by a pair $(\sigma,l)$, where $\sigma$ is a memory store and $l\in [1,n]$ is the program point of the next instruction to be executed. We denote by $State=\Sigma\times \mathbb{N}$ the set of all possible program states.
\end{definition}

\noindent We provide a big-step semantics using transition relation $\trans_{\pi}\:\subseteq State\times State$, which specifies how a single instruction of a program $\pi$ affects its state. Our description relies on relation $\Downarrow\subseteq(\Sigma\times Expr)\times \mathbb{Z}$ to describe how expressions are evaluated in a given memory store.


\begin{definition}[Big-Step Transitions]
%For any program $\pi$, relation $\Rightarrow_{\pi}\subseteq State\times State$ is defined as follows, with meta-variables $\texttt{x}, \texttt{y}\in Var$, $\texttt{e}\in Expr$, and $\texttt{m}\in Num$:
For any program $\pi$, we define relation $\Rightarrow_{\pi}\:\subseteq State\times State$ as follows, with meta-variables $\texttt{x}, \texttt{y}\in Var$, $\texttt{e}\in Expr$, and $\texttt{m}\in Num$:

\begin{small}

% asgn
\begin{equation}
\label{eq:asgn-sem}
\frac
{I_l=\texttt{x:=e} ~~ \wedge ~~ (\sigma, \texttt{e}) \Downarrow v}
{(\sigma, l)\Rightarrow_{\pi} (\sigma[\wx\gets v], l+1)}
\end{equation}
% if (0)
\vspace{0.5mm}
\begin{equation}
\label{eq:ifz-sem}
\frac
{I_l=\texttt{if (e) goto m} ~~ \wedge ~~ (\sigma, \texttt{e}) \Downarrow 0}
{(\sigma, l)\Rightarrow_{\pi} (\sigma, l+1)}
\end{equation}
% if (!0)
\vspace{0.5mm}
\begin{equation}
\label{eq:ifnz-sem}
\frac
{I_l=\texttt{if (e) goto m} ~~ \wedge ~~ (\sigma, \texttt{e}) \Downarrow v ~~~ \wedge ~~~ v\neq 0}
{(\sigma, l)\Rightarrow_{\pi} (\sigma, \texttt{m})}
\end{equation}
% goto
\vspace{0.5mm}
\begin{equation}
\label{eq:goto-sem}
\frac
{I_l=\texttt{goto m}}
{(\sigma, l)\Rightarrow_{\pi} (\sigma, \texttt{m})}
\end{equation}
% skip
\vspace{0.5mm}
\begin{equation}
\label{eq:skip-sem}
\frac
{I_l=\texttt{skip}}
{(\sigma, l)\Rightarrow_{\pi} (\sigma, l+1)}
\end{equation}
% in
\vspace{0.5mm}
\begin{equation}
\label{eq:in-sem}
\frac
{I_1=\texttt{in x y}~\cdots ~~ \wedge ~~~ \sigma(\texttt{x})\neq\bot ~~~ \wedge ~~~ \sigma(\texttt{y})\neq\bot ~~~ \wedge ~~~ \cdots }
{(\sigma, 1)\Rightarrow_{\pi} (\sigma, 2)}
\end{equation}
% out
%\vspace{0.5mm}
\begin{equation}
\label{eq:out-sem}
\frac
{I_n=\texttt{out x y}~\cdots ~~ \wedge ~~~ \sigma(\texttt{x})\neq\bot ~~~ \wedge ~~~ \sigma(\texttt{y})\neq\bot ~~~ \wedge ~~~ \cdots }
{(\sigma, n)\Rightarrow_{\pi} (\sigma\vert_{\{\texttt{x}, \texttt{y}, \cdots\}}, n+1)}
\end{equation}

\end{small}

\end{definition}

\noindent For a transition to apply, we implicitly assume that $I_l$ is defined, i.e., $l\in[1,n]$. 
\ifx\noauthorea\undefined
Notice that we intentionally do not provide any transition rule for {\tt abort} instructions, providing explicit means to let a program have undefined semantics. This might be useful in supporting unsound speculative optimizations.
\fi

\begin{definition}[Program Semantic Function]
\label{de:program-semantics}
%The semantic function $\mysem{\pi}:\Sigma \rightarrow \Sigma$ of a program $\pi$ is defined as:
We define the semantic function $\mysem{\pi}:\Sigma \rightarrow \Sigma$ of a program $\pi$ as: 
\begin{gather*}
\forall \sigma\in\Sigma: ~~ \mysem{\pi}(\sigma)=\sigma'~~%\vert_{\{\wx\,:\,I_{n}={\tt out}~\cdots~\wx~\cdots\}} ~~ \\ 
\Longleftrightarrow ~~ (\sigma,1) \Rightarrow^{*}_{\pi} (\sigma',|\pi|+1)
\end{gather*}
where $\Rightarrow^{*}_{\pi}$ is the transitive closure of $\Rightarrow_{\pi}$.
\end{definition}

\noindent Note that a program has undefined semantics if its execution on a given store does not reach the final \texttt{out} instruction. This accounts for infinite loops, abort instructions, exceptions, and ill-defined programs or input stores. 

We define the notion of program semantic equivalence as follows:

\begin{definition}[Program Equivalence]
\label{de:semantic-equivalence}
Two programs $\pi_1$ and $\pi_2$ are {\em semantically equivalent} iff $\mysem{\pi_1}=\mysem{\pi_2}$.
\end{definition}

\noindent A notion that will be useful in proving correctness in our framework is that of {\em trace} of a transition system:

\begin{definition}[Traces]
\label{de:exec-trace}
A {\em trace} in a transition system $(S,$ $R\subseteq S^2)$ starting from $s\in S$ is a sequence $\tau=\langle s_0,s_1,\ldots,$ $s_i,\ldots\rangle$ such that $s_0=s$ and $\forall i\ge 0:~s_i\in\tau ~ \wedge ~ s_i~R~s_{i+1}$ $\Longleftrightarrow s_{i+1}\in\tau$. By ${\cal T}_{R,s}$ we denote the system of all traces of $(S,R\subseteq S^2)$ starting from $s$. By $\tau[i]$ we denote the $i$-th state of $\tau$, i.e., $\tau[i]=s_i$. Furthermore, if trace $\tau$ is finite then $|\tau|$ denotes the index of its final state, i.e., $\tau=\langle s_0,s_1,\ldots,s_{|\tau|}\rangle$, otherwise $|\tau|=\infty$. Finally, $dom(\tau)=\{i:s_i\in\tau\}$ denotes the set of indices of states in $\tau$.
\end{definition}

\noindent Notice that, since $\Rightarrow_{\pi}$ is deterministic in our language, then for any initial store $\sigma$, the system of traces ${\cal T}_{\Rightarrow_{\pi},(\sigma,1)}$ of the execution transition system $(Store,\Rightarrow_{\pi})$ contains a single trace, which we denote by $\tau_{\pi\sigma}$.

Finally, we provide a formal definition of control flow graph, which will be useful in defining computation tree logic operators for reasoning on program properties:

\begin{definition}[Control Flow Graph]
\label{de:cfg}
%The {\em control flow graph} G for a program $\pi=\langle I_1, I_2, \ldots, I_n \rangle$ is described by a pair $(V,E)$ where $V = \{ I_1, I_2, \ldots, I_n \}$ and $E = \{(I_i, I_{i+1})\:|\: I_i \neq \textsf{abort} \wedge I_i \neq \textsf{goto m}, \!\textsf{ m}\in Num \}\;\cup\;\{(I_i, I_m)\:|\: I_i = \textsf{goto m} \vee I_i = \textsf{if (e) goto m}, \!\textsf{ m}\in Num, \!\textsf{ e}\in Expr \}$.
%The {\em control flow graph} G for a program $\pi=\langle I_1, I_2, \ldots, I_n \rangle$ is described by a tuple $\langle V, I: V\rightarrow Num, E \subseteq V\times V\rangle$ where:
The {\em control flow graph} $G$ for a program $\pi=\langle I_1, I_2, \ldots, I_n \rangle$ is described by a pair $(V, E \subseteq V\times V)$ where:
\begin{equation*}
\begin{align}
V &= \{ I_1, I_2, \ldots, I_n \} \\
%I(v) &= \{ i | v = I_i, v\in V\} \\
E &= \{(I_i, I_{i+1})\:|\: I_i \neq \textsf{abort} \wedge I_i \neq \textsf{goto m}, \!\textsf{ m}\in Num \} \\
&\cup\;\{(I_i, I_m)\:|\: I_i = \textsf{goto m} \vee I_i = \textsf{if (e) goto m}, \!\textsf{ m}\in Num, \!\textsf{ e}\in Expr \}.
\end{align}
\end{equation*}
%\noindent We also define an auxiliary function $I: V\rightarrow Num$ that returns the $i$-th index in $\pi$ of an instruction $v\in V$.
\end{definition}

% !TEX root = thesis.tex

\subsection{Program Properties and Transformations}
\label{ss:osr-reasoning-and-transformations}

In this section we present a formalism based on {\em computation tree logic} (CTL) to reason about program properties and describe program transformations through rewrite rules with side conditions~\cite{Clarke86,Lacey04,Kalvala09}.

\subsubsection*{Reasoning about Program Properties}

To analyze properties of a program, we use Boolean formulas with free meta-variables that combine facts that must hold globally or at certain points of a program. Formulas can be checked against concrete programs by a {\em model checker}. For any program $\pi$ and formula $\phi$, the checker verifies whether there exists a substitution $\theta$ that binds free meta-variables with program objects so that $\theta(\phi)$ is satisfied in $\pi$. 
In this thesis, by $\cal{A}\models \phi$ we mean that $\phi$ is true in $\cal{A}$, i.e., formula $\phi$ is satisfied by structure $\cal{A}$ (or equivalently, $\cal{A}$ models $\phi$)~\cite{Clarke86}. %We use notation $\cal{A}\models_{\theta} \phi$ as a shortcut for $\cal{A}\models \theta(\phi)$. 

\noindent Two global predicates that we will use later on are ${\tt conlit}(\wc)$, which states that an expression $\wc$ is a constant literal, and ${\tt freevar}(\wx,\we)$, which holds if and only if $\wx$ is a free variable of expression $\we$.

%To support analyses based on facts that involve finite maximal paths in the control flow graph (CFG), such as liveness and dominance, we use formulas based on {\em computation tree logic} (CTL) operators~\cite{Clarke86,Lacey04,Kalvala09}. In order to introduce these operators, we need to formalize the concept of finite maximal paths first.
To support analyses based on facts that involve finite maximal paths in the control flow graph (CFG), such as liveness and dominance, we use formulas based on CTL operators. In order to introduce these operators, we need to formalize the concept of finite maximal paths first.

\begin{definition}[Set of Complete Paths] Given a control flow graph $G=(V,E)$ and an initial node $n_0\in V$, the {\em set of complete paths} $CPaths(n_0,G)$ starting at $n_0$ consists of all finite sequences $\langle n_0,n_1,\ldots,n_k\rangle$ such that $(n_i,n_{i+1})\in E$ for all $n_i$ with $i<k$, and such that there does not exist a $n_{k+1}$ such that $(n_k,n_{k+1})\in E$.
\end{definition}

\noindent Complete paths from a specified node (i.e., instruction) are thus maximal finite sequences of connected nodes through a control flow graph from an initial point to a sink node, which in our setting is unique (unless {\tt abort} instructions are present) and corresponds to the final instruction $I_n$.

First-order CTL can be used to specify properties of nodes and paths in a CFG. In particular, temporal CTL operators can be used to express properties of some or all possible future computational paths, any one of which might be an actual path that is realized. Before formalizing the temporal operators that we are going to use in the remainder of this chapter, we provide an intuitive definition for them. We say that, given a point $l$ in a program $\pi$ and two formulas $\phi$ and $\psi$, the following predicates are satisfied at $l$ if:

\begin{itemize}[parsep=0pt,topsep=3pt]
\item $\overrightarrow{AX}(\phi)$: $\phi$ holds for all immediate successors of $l$;
\item $\overrightarrow{EX}(\phi)$: $\phi$ holds for at least one immediate successor of $l$;
\item $\overrightarrow{A}(\phi~U~\psi)$: $\phi$ holds on all paths from $l$, until $\psi$ holds;
\item $\overrightarrow{E}(\phi~U~\psi)$: $\phi$ holds on at least one path from $l$, until $\psi$ holds.
\end{itemize}
\noindent Corresponding operators $\overleftarrow{AX}$ and $\overleftarrow{EX}$ are defined for immediate predecessors of $l$, while $\overleftarrow{A}$ and $\overleftarrow{E}$ refer to backward paths from $l$.

%\begin{definition}[Until Predicate]
%Given a node $n_0$ in the control flow graph $G$ and a path $p = \langle n_0,n_1,\ldots,n_k\rangle \in CPaths(n_0,G)$, we say that the predicate {\em Until}$(p,\phi,\psi)$ holds if:
%\begin{equation*}
%\exists j: 0 \le j\le k: n_j \models \psi \; \wedge \forall 0 \le i < j: n_i \models \phi
%\end{equation*}
%\end{definition}

\begin{definition}[Temporal Operators]
Given a node $n$ in the control flow graph $G=(V,E)$ of a program $\pi$, we define the following CTL {\em temporal operators} as:

%\begin{equation*}
\begin{align*}
n \models \overrightarrow{AX}(\phi) &\Longleftrightarrow \forall m: (n,m)\in E: \pi,m\models\phi \\
n \models \overrightarrow{EX}(\phi) &\Longleftrightarrow \exists m: (n,m)\in E: \pi, m\models\phi \\
n \models \overrightarrow{A}(\phi~U~\psi) &\Longleftrightarrow \forall p: p\in CPaths(n,G): Until(\pi, p,\phi,\psi) \\
n \models \overrightarrow{E}(\phi~U~\psi) &\Longleftrightarrow \exists p: p\in CPaths(n,G): Until(\pi, p,\phi,\psi) \\
\end{align*}
%\end{equation*}

%\vspace{0.5em}
\vspace{-0.5em}
\noindent where predicate $Until(\pi,p,\phi,\psi)$ holds for $p = \langle n_0,n_1,\ldots,n_k\rangle \in CPaths(n_0,G)$ if:
\vspace{-0.5em}

\begin{equation*}
\exists j: 0 \le j\le k: \pi, n_j \models \psi \; \wedge \: \forall 0 \le i < j: \pi, n_i \models \phi
\end{equation*}

\noindent Operators $\overleftarrow{AX}$, $\overleftarrow{EX}$, $\overleftarrow{A}$, and $\overleftarrow{E}$ can be defined similarly on the reverse control flow graph $\overleftarrow{G}$, which is identical to $G$ but with every edge in $\overleftarrow{E}$ flipped.
\end{definition}

\noindent Operators $A$ and $E$ are quantifiers over paths, while $X$ and $U$ path-specific quantifiers. Notice that $\phi~U~\psi$ requires that $\phi$ has to hold at least until at some node $\psi$ is satisfied: this implies that $\psi$ will be verified in the future.

\myfigure\ref{fig:osr-loc-pred} shows a number of local predicates that will be useful throughout this thesis. For instance, $\pi,l\models \ureachdef(\wx, l')$ ({\em unique reaching definition}) holds if and only if variable $\wx$ is defined at $l$ and on all paths in the control flow graph starting from an immediate successor of $l$, $\wx$ is not redefined until point $l'$ is reached, i.e., there is a unique definition of $\wx$ that reaches $l'$, and this definition is at $l$. \ureachdef's formulation relies on nested CTL operators: $\overrightarrow{AX}$ is used to encode a property for all successors of $l$, while the nested $\overrightarrow{A}$ captures all forward paths starting at such nodes.

The following definition will be useful, too:

\begin{definition}[Live Variables]
\label{de:live-var}
The set of live variables of a program $\pi$ at point $l$ is defined as:
\vspace{-1mm}
\begin{equation*}
\live(\pi,l) \triangleq \{ ~ \wx\in Var ~ | ~ \pi, l\models \islive(\wx) ~ \}
\end{equation*}
\end{definition}

%% THIS EXAMPLE IS NOT CORRECT AS WE HAVE DEFINED use() OVER VARIABLES ONLY!!!
%\begin{example}
%Available expression analysis is a forward data-flow problem. For each point in a program, an algorithm determines the set of of expressions that do not need to be recomputed. Given a program $\pi$ and an instruction $p$, we can check whether an expression $e$ is available at it using:

%\begin{equation*}
 %\pi, p \models \overleftarrow{A}(\wtrans(e)~U~\wuse(e))
%\end{equation*}

%\noindent which captures the idea that for all backward starting at $p$, a calculation of $e$ is reached before any of its constituents is possibly modified.
%\end{example}

\begin{example}
Dominance analysis is widely employed in a number of program analysis and optimization techniques. In a CFG, we say that a node $n$ dominates another node $m$ if every path from the CFG's entry node to $m$ must go through $n$. Using CTL operators, we can easily encode this property. Given a program $\pi$ as in \mydefinition\ref{de:program}, we can write:
\begin{equation*}
 \mytt{dom}(n,m) \Longleftrightarrow \pi,I_1 \models \neg E(\neg\wpoint(n)~U~\wpoint(m))
\end{equation*}

\noindent which captures the idea that there does not exist a path starting at the entry node (i.e, the first instruction in $\pi$) that reaches $m$ without reaching $n$ first.
\end{example}

\begin{figure}[!ht]
\vspace{-3mm}
\begin{small}
\begin{eqnarray*}
\wdef(\wx) & \triangleq & I_l= \texttt{x:=e} ~~ \vee ~~ I_l= \texttt{in} ~ \cdots ~ \texttt{x} \cdots \\
%                            &            & I_l= \texttt{in} ~ \cdots ~ \texttt{x} \cdots \\
                            &            & [\wx ~ \textit{is defined by instruction} ~ I_l ~ \textit{in} ~ \pi] \\
\wuse(\wx) & \triangleq & I_l= \texttt{y:=e[x]} ~ \vee  \\
                            &            & I_l= \texttt{if (e[x]) goto m} ~ \vee \ \\
                            &            & I_l= \texttt{out} ~ \cdots ~ \texttt{x} \cdots \\
                            &            & [\wx ~ \textit{is used by instruction} ~ I_l ~ \textit{in} ~ \pi] \\
\wtrans(\we) & \triangleq & I_l= \texttt{x:=e'} ~ \wedge ~ \neg\wfreevar(\wx,\we) ~ \vee \\
                            &            & I_l\neq\texttt{x:=e'} \\
%                            &            & [\we ~ \textit{is not affected by instruction} ~ I_l ~ \textit{in} ~ \pi] \\
                            &            & [\textit{no constituent of}~\we~\textit{is modified by instruction} ~ I_l ~ \textit{in} ~ \pi] \\
\islive(\wx) & \triangleq & \overleftarrow{AX}\overleftarrow{A}(\text{true} ~ U ~ \wdef(\wx)) ~ \wedge \\
                            &            & \overrightarrow{E}(\neg\wdef(\wx) ~ U ~ \wuse(\wx)) \\
                            &            & [\wx ~ \textit{is live at program point} ~ l ~ \textit{in} ~ \pi] \\
\ureachdef(\wx,l') & \triangleq & \overleftarrow{AX}\overleftarrow{A}(\neg\wdef(\wx) ~ U ~ \point(l')\wedge\wdef(\wx)) \\
                            &            & [\textit{unique definition of}~\wx~{at}~l'~\textit{reaching}~l~\textit{in} ~ \pi] \\
\stmt(I) & \triangleq & I=I_l ~~~ [I ~ \textit{is the instruction at} ~ l ~ \textit{in} ~ \pi]\\
%                 &            & [I ~ \textit{is the instruction at} ~ l ~ \textit{in} ~ \pi] \\
\point(\texttt{m}) & \triangleq & \texttt{m}=l ~~~ [\textit{program point} ~ \wm ~ \textit{is} ~ l ~ \textit{in} ~ \pi]
%                           &            & [\textit{program point} ~ \wm ~ \textit{is} ~ l ~ \textit{in} ~ \pi]
\end{eqnarray*}
\vspace{-4mm}
\end{small}
\caption{\label{fig:osr-loc-pred}Predicates expressing local properties of a point $l\in [1,n]$ in a program $\pi=\langle I_1,\ldots,I_n\rangle$, with meta-variables $\texttt{e},\texttt{e'}\in Expr$, $\texttt{x}, \texttt{y}\in Var$, and $l, \texttt{m}\in Num$.}
\end{figure}

\subsubsection*{Program Transformations}

To describe program transformations, we use rewrite rules with side conditions in a similar manner to~\cite{Lacey04,Kundu09}. We consider generalized rules that transform multiple instructions simultaneously, with side conditions drawn from CTL formulas:

\begin{definition}[Rewrite Rule]
\label{de:rewrite-rule}
A rule $T$ has the form:
\vspace{-1mm}
\begin{equation*}
\begin{array}{lllll}
T = & m_1: \hat{I}_1 \Longrightarrow \hat{I'}_1 %& m_2: I_2 \Longrightarrow I'_2
& \cdots
& m_r: \hat{I}_r \Longrightarrow \hat{I'}_r
& {\tt if} ~ \phi
\end{array}
\vspace{-1mm}
\end{equation*}
\noindent where $\forall k\in[1,r]$, $m_k$ is a meta-variable that denotes a program point, $\hat{I}_k$ and $\hat{I'}_k$ are program instructions that can contain meta-variables, and $\phi$ is a side condition that states whether the rewriting rule can be applied to the input program. We denote by $\Tau$ the set of all possible rewrite rules.
\end{definition}

\noindent An elementary example of rewrite rule with meta-variables $\wm$, $\wx$, and $\wy$ is: $$m: ~ {\tt y:=2*x} ~~ \Longrightarrow ~~ {\tt y:=x+x} ~~~ {\tt if} ~ true$$ which implements a peephole optimization based on a weak form of operator strength reduction~\cite{Cooper01}.

\noindent Rules can be applied to concrete programs by a transformation engine based on model checking: when the checker finds a substitution $\theta$ that binds free meta-variables with program objects so that $\theta(\phi)$ is satisfied in $\pi$ and $\theta(\hat{I}_k)=I_{\theta(m_k)}\in \pi$ for some $k\in[1,t]$, then $I_{\theta(m_k)}$ is replaced with $\theta(\hat{I'}_k)=I'_{\theta(m_k)}\in \pi'$, as formalized next:

\begin{definition}[Rule Semantics]
\label{de:trans-func}
Let $T$ be a rewrite rule as in \ref{de:rewrite-rule}. Transformation function $\mysem{T}: Prog\rightarrow Prog$ is defined as follows:
\vspace{-2mm}
\begin{multline*}
\forall \pi, \pi'\in Prog: \pi'=\mysem{T}(\pi) \Longleftrightarrow
\exists ~ \theta: ~ \pi\models \theta(\phi) ~ \wedge ~ \\
\forall k\in[1,r]: \theta(\hat{I}_k)=I_{\theta(m_k)}\in \pi ~ \wedge ~ \theta(\hat{I'}_k)=I'_{\theta(m_k)}\in \pi'
\end{multline*}
\end{definition}

\noindent In this thesis, we focus on transformations that do not alter the semantics of a program:

\begin{definition}[Semantics-Preserving Rules]
\label{de:sound-trans}
A rewrite rule $T$ is {\em semantics-preserving} if for any program $\pi$ it holds $\mysem{\pi}=\mysem{\pi'}$, where $\pi'=\mysem{T}(\pi)$.
\end{definition}

\noindent Examples of semantics-preserving rules for classic compiler optimizations (as proven in~\cite{Lacey02,Lacey04}) are given in \myfigure\ref{fig:sample-trans}.

\begin{figure}[ht]
\begin{center}
\begin{small}

\begin{tabularx}{0.8\textwidth}{|X|}\hline
{\bf Constant propagation} (CP):\\\hline
$m: ~ {\tt x:=e[v]} ~~ \Longrightarrow ~~ {\tt x:=e[c]}$ \\
${\tt if} ~~ \wconlit(\wc) ~ \wedge ~ m \models \overleftarrow{A}(\neg\wdef(\wv) ~ U ~ \wstmt({\tt v:=c}))$ \\\hline
\end{tabularx}

\vspace{2mm}

\begin{tabularx}{0.8\textwidth}{|X|}\hline
{\bf Dead code elimination} (DCE):\\\hline
$m: ~ {\tt x:=e} ~~ \Longrightarrow ~~ {\tt skip}$ \\
${\tt if} ~~ m \models \overrightarrow{AX} ~ \neg\overrightarrow{E}(true ~ U ~ \wuse(\wx))$ \\\hline
\end{tabularx}

\vspace{2mm}

\begin{tabularx}{0.8\textwidth}{|X|}\hline
{\bf Code hoisting} (Hoist):\\\hline
$p: ~ {\tt skip} ~~ \Longrightarrow ~~ {\tt x:=e}$ \\
$q: ~ {\tt x:=e} ~~ \Longrightarrow ~~ {\tt skip}$ \\
${\tt if} ~~ p \models \overrightarrow{A}(\neg\wuse(\wx) ~ U ~ \wpoint(q)) ~~ \wedge$ \\
$\hphantom{\texttt{if}} ~~ q \models \overleftarrow{A}((\neg\wdef(\wx)\vee\wpoint(q))\wedge \wtrans(e) ~ U ~ \wpoint(p))$ \\\hline
\end{tabularx}

\vspace{-2mm}

\end{small}
\end{center}
\caption{\label{fig:sample-trans} Rewriting rules for defining CP, DCE, and Hoist transformations.}
\end{figure}

The constant propagation (CP) rule replaces uses of a variable $v$ at a node $m$ with a constant $c$. Its side condition is satisfied when in all backward paths starting at $m$, the first definition of $v$ we encounter is always $v:=c$.

The dead code elimination (DCE) rule deletes an instruction at a node $m$ if the result of its computation will never be used later in the program's execution. As we are not interested in uses of the variable itself at $m$, in the side condition we skip past it with $AX$ and specify that there should not exist a forward path that eventually uses (i.e., reads from) the variable. 

Finally, the code hoisting (Hoist) rule moves an assignment of the form $x:=v[e]$ from a node $q$ to a node $p$ provided that two conditions are met. The first requires that in all forward paths starting at the insertion point $p$, $x$ is not used until the original location $q$ is reached. The second requires that in all backward paths starting at $q$, $x$ is not reassigned at any node other than $q$ and the constituents of $e$ are not redefined, until the insertion point $p$ is reached. %Intuitively, \wtrans$(e)$ is used to ensure that $e$ is available at $q$ after the transformation has been applied.