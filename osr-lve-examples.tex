% !TEX root = thesis.tex

\subsubsection{Examples of LVE Transformations}

\label{ss:lve-trans}

In this section, we show that classic compiler optimizations such as constant propagation, dead code elimination, and code hoisting as defined in \myfigure\ref{fig:sample-trans} are all examples of live-variable equivalent transformations. Hence, they are provably correct building blocks of an OSR-aware compilation toolchain based on algorithm $\osrtrans$. These optimizations are representatives of a broad class of transformations that insert, delete, and modify instructions. Further optimizations, which we do not formally discuss here, are evaluated in \mysection\ref{se:eval-OSR-alC}.

\begin{theorem}
\label{th:lve-trans-examples}
Transformations CP, DCE, and Hoist of \myfigure\ref{fig:sample-trans} are live-variable equivalent.
\end{theorem}

\begin{myproof}
In \cite{Lacey02}, CP, DCE, and Hoist are proven correct, each using a different bisimulation relation $R$. 
%We propose a unifying approach based on live-variable bisimilarity.
For CP, $R$ is simply the identity relation, hence $A(l)=Val\supseteq \live(\pi,l)\cap \live(\pi',l)$ in \mydefinition\ref{de:state-equiv-relation}.

For the other two transformations, $R$ is piecewise-defined on the indices of the traces. For any initial store $\sigma\in \Sigma$, let $\tau_{\pi\sigma}[i]=(\sigma_i,l_i)$, $\tau_{\pi'\sigma}[i]=(\sigma'_i,l'_i)$, and $t$ be the index of the final state in both traces (note that $|\tau_{\pi\sigma}|=|\tau_{\pi'\sigma}|$ from \ref{le:bisim-prop}). Let also $\theta$ be a substitution that bounds free meta-variables with concrete program objects so that a rule's side-condition is satisfied.

For DCE, $R$ is the identity relation before the eliminated assignment \mytt{x:=e}, and $A(l)=Val\setminus\{\theta(\wx)\}=\live(\pi,l)\cap \live(\pi',l)$ after it. $R$ is a bisimulation such that $\forall i\in[1,t]$ $l_i=l'_i$ and both the following conditions hold:
\begin{enumerate}
 \item $[\forall j, j<i \Rightarrow l_j\neq \theta(p)] \Rightarrow \sigma_i=\sigma'_i~$ and
 \item $[\exists j, j\leq i \wedge l_j = \theta(p)] \Rightarrow \sigma_i\setminus\texttt{x}=\sigma'_i\setminus\texttt{x}$
\end{enumerate}
\noindent where $p$ is the meta-variable for the eliminated assignment in $\pi'$, and $\sigma\setminus\texttt{x}$ is syntactic sugar for $\sigma\vert_{D(\sigma)}$, where $D(\sigma)=\{v\in Var ~|~ v\neq \texttt{x} ~\wedge~ \sigma(v)\neq\bot\}$ is the set of all the variable identifiers other than \mytt{x} currently defined in $\sigma$.

For Hoist, $R$ is the identity relation before $\theta(p)$ and after $\theta(q)$ (see \myfigure\ref{fig:sample-trans}), and $A(l)=Val\setminus\{\theta(\wx)\}=\live(\pi,l)\cap \live(\pi',l)$ between them. Formally, we have that $\forall i\in[1,t]$ $l_i=l'_i$ and one of the following cases holds:
\begin{enumerate}
 \item $\sigma_t=\sigma'_t ~\wedge~ \forall i~[0\leq i<t ~\Rightarrow~ l_i \notin \{\theta(p),~\theta(q)\}]$
 \item $\begin{aligned}[t]
\sigma_t=\sigma'_t ~\wedge~ \exists i~[&0\leq i<t ~\wedge~ l_i=\theta(q) ~\wedge~ \sigma_i=\sigma'_i ~\wedge\\
&\forall j~(i<j<t ~\Rightarrow~ l_j \notin \{\theta(p),~\theta(q)\})]
       \end{aligned}$
 \item $\begin{aligned}[t] 
\exists i~[0\leq i<t ~\wedge~ &l_i=\theta(p) ~\wedge~ (\sigma_t\setminus\texttt{x}=\sigma_t'\setminus\texttt{x}) ~\wedge~ (\sigma_i\setminus\texttt{x}=\sigma_i'\setminus\texttt{x}) ~\wedge\\
&\forall j~(i<j<t ~\Rightarrow~ l_j \notin \{\theta(p),~\theta(q)\}]
       \end{aligned}$
\end{enumerate}

\noindent Case 1 applies before $\theta(p)$ is reached in the trace. Case 3 applies after $\theta(p)$ has been reached, but $\theta(q)$ has not. Finally, case 2 applies after $\theta(q)$ has been reached.
\end{myproof}