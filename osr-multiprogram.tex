% !TEX root = thesis.tex

\subsection{Multi-Version Programs}

In this section, we propose a general OSR model where computations are described by a {\em multi-version program}, which consists of different versions of a program along with OSR mappings that allow execution to be transferred between them.

\begin{definition}[Multi-Version Program]
\label{de:mv-program}
A multi-version program is modeled by an edge-labeled graph $\Pi=({\cal V}, {\cal E}, {\cal M})$ where ${\cal V}=\{ \pi_1, \pi_2, \ldots,\pi_r\}$ is a set of program versions, ${\cal E}\subseteq \Pi^2$ is a set of edges such that $(\pi_p,\pi_q)$ indicates that an OSR transition can be fired from some point of $\pi_p$ to $\pi_q$,
%any point in $dom({\cal M}(\pi_i,\pi_j))$, 
and ${\cal M}:{\cal E}\rightarrow OSRMap$ labels each edge $(\pi,\pi')\in {\cal E}$ with an OSR mapping from $\pi$ to $\pi'$.
\end{definition}

\subsubsection*{Semantics}

The state of a multi-version program is similar to the state of a program (\ref{de:prog-state}), but it also includes the index of the currently executed program version:

\begin{definition}[Multi-Version Program State]
The {\em state} of a multi-version program $\Pi=({\cal V}, {\cal E}, {\cal M})$ is described by a triple $(p,\sigma,l)$, where $p\in[1,|{\cal V}|]$ is the index of a program version, $\sigma$ is a memory store, and $l\in [1,|\pi_p|]$ is the point of the next instruction to be executed in $\pi_p$. The {\em initial state} from a store $\sigma$ is $(1,\sigma,1)$, i.e., computations start at $\pi_1$. We denote by $MState=\mathbb{N}\times\Sigma\times \mathbb{N}$ the set of all possible multi-version program states.
\end{definition}

\noindent The execution semantics of a multi-version program is described by the following transition relation:

\begin{definition}[Multi-Version Big-Step Transitions]
\label{de:osr-semantics}
For any multi-version program $\Pi$, relation $\Rightarrow_{\Pi}\subseteq MState\times MState$ is defined as follows:%, with meta-variables $\texttt{x}, \texttt{y}\in Var$, $\texttt{e}\in Expr$, and $\texttt{m}\in Num$:

\begin{footnotesize}
\begin{equation}
\label{eq:mv-big-step}
\begin{array}{rc}
(Norm)
&
\dfrac
{(\sigma, l)\Rightarrow_{\pi_p} (\sigma',l')}
{(p,\sigma, l)\Rightarrow_{\Pi} (p,\sigma',l')}
\\
\\
(OSR)
&
\dfrac
{(\pi_p,\pi_q)\in {\cal E} ~ \wedge ~ (l',\chi)={\cal M}(\pi_p,\pi_q)(l) ~ \wedge ~ \sigma'=\mysem{\chi}(\sigma)}
{(p,\sigma, l)\Rightarrow_{\Pi} (q,\sigma',l')}\\
\end{array}
\end{equation}
\end{footnotesize}
\end{definition}

\noindent The meaning is that at any time, execution can either continue in the current program version (Norm rule), or an OSR transition -- if possible at the current point -- can direct the control to another program version (OSR rule). The choice is non-deterministic, i.e., an oracle can tell the execution engine which rule to apply.

In practice, the choice may be based for instance on profile data gathered by the runtime system: a common strategy is to dynamically ``OSR'' to the available version with the best expected performance on the actual workload. Notice that since $\Rightarrow_{\Pi}$ may be non-deterministic, in general there may be different final stores for the same initial store. However, we are only interested in multi-version programs that deterministically yield a unique result, which guarantees semantic transparency of OSR transitions. 

To characterize the execution behavior of a multi-version program, we consider the system of traces of an execution transition system that start from a given initial state.

\begin{definition}[Trace System of Multi-Version Program]
\label{de:mvp-exec-system}
The system of traces ${\cal T}_{\Pi,\sigma}$ contains all traces $\tau$ of transition system $(MState,\Rightarrow_{\Pi})$ such that $\tau[0]=(1,\sigma,1)$.
\end{definition}

\begin{definition}[Deterministic Multi-Version Program]
\label{de:deterministic-mvp}
A multi-version program $\Pi$ is deterministic iff $\forall \sigma\in\Sigma$, either all traces in ${\cal T}_{\Pi,\sigma}$ are infinite, or they all lead to the same store, i.e.:
\begin{gather*}
\forall \tau, \tau'\in{\cal T}_{\Pi,\sigma}: ~~ \big(|\tau|=\infty ~ \Longleftrightarrow ~ |\tau'|=\infty\big) ~ \wedge \\
\big(|\tau|<\infty ~ \Longrightarrow ~ \exists~p,p',l,l'\in\mathbb{N}, \sigma,\sigma'\in\Sigma: ~ \\
\tau[|\tau|]=(p,\sigma,l) ~ \wedge ~ \tau'[|\tau'|]=(p',\sigma',l') ~ \wedge ~ \sigma=\sigma' \big).
\end{gather*}
\end{definition}

\noindent The meaning of a deterministic multi-version program can be defined as follows:

\begin{definition}[Multi-Version Semantic Function]
\label{de:mv-program-semantics}
The semantic function $\mysem{\Pi}:\Sigma \rightarrow \Sigma$ of a deterministic multi-version program $\Pi$ is defined as: 
$$
%\forall \sigma\in\Sigma: ~~ \mysem{\Pi}(\sigma)=\sigma' ~~ \Longleftrightarrow ~~ (1,\sigma,1) \Rightarrow^{*}_{\Pi} (p,\sigma',|\pi_p|+1) ~ \}
\forall \sigma\in\Sigma: ~~ \mysem{\Pi}(\sigma)=\sigma' ~~ \Longleftrightarrow ~~ (1,\sigma,1) \Rightarrow^{*}_{\Pi} (p,\sigma',|\pi_p|+1)
$$
where $\Rightarrow^{*}_{\Pi}$ is the transitive closure of $\Rightarrow_{\Pi}$.
\end{definition}

\subsubsection*{Generation Algorithm and Correctness}

A natural way to generate a multi-version program consists in starting from a base program $\pi_1$ and constructing a tree of different versions, where each version is derived from its parent by applying one or more transformations. Using this approach and procedure \dopasses\ described in \mysection\ref{ss:trans-compose}, it is straightforward to construct a multi-version program $\Pi=({\cal V}, {\cal E}, {\cal M})$ such that:
\vspace{-1mm}
\begin{align*}
(\pi_p,\pi_q)\in {\cal E} ~~ \Longleftrightarrow ~~ \exists L: ~ &{\tt do\_passes}(\pi_p,L)=(\pi_q,\mu,\mu') ~ \wedge ~ {\cal M}(\pi_p,\pi_q)=\mu ~~ \vee \\
&{\tt do\_passes}(\pi_q,L)=(\pi_p,\mu,\mu') ~ \wedge ~ {\cal M}(\pi_p,\pi_q)=\mu'
\end{align*}

\noindent To prove the correctness of this approach, we introduce a preliminary lemma and then use it to prove that a multi-version program built in this way is deterministic.

\begin{lemma}
\label{le:comp-lemma}
Let $\tau\in{\cal T}_{\Pi,\sigma}$ be an execution trace in the system of the traces for the multi-version program $\Pi$ $=({\cal V}, {\cal E}, {\cal M})$ constructed using \dopasses\ and LVE transformations, and let $\omega_1,\ldots,\omega_k$ be the indices of $\tau$ where an OSR transition has just occurred, with $\tau[\omega_i]=(p_{\omega_i}, \sigma_{\omega_i}, l_{\omega_i})$. Then $\forall i\in[1,k]$ there exists a state $(\hat{\sigma}_i,\hat{l}_i)$ in the trace of $\pi_{p_{\omega_{i}}}$ starting from the initial store $\sigma$ such that $\hat{l}_i=l_{\omega_i}$ and $\hat{\sigma}_i\vert_{\live(\pi_{p_{\omega_{i}}},\,\hat{l}_i)}=\sigma_{\omega_i}\vert_{\live(\pi_{p_{\omega_{i}}},\,\hat{l}_i)}$.
\end{lemma}
\begin{myproof}
To simplify the notation we introduce:
\begin{equation*}
\hat{\pi}_i = \begin{cases}
\pi_1 & \text{if } i=0\\
\pi_{p_{\omega_i}} & \text{if } i \in [1,k]
\end{cases}
\end{equation*}

\noindent From \ref{eq:mv-big-step} we can write that $\tau[\omega_i]=(p_{\omega_i}, \sigma_{\omega_i},l_{\omega_i})$ has been obtained from $\tau[\omega_i-1]=(p_{\omega_i-1}, \sigma_{\omega_i-1}, l_{\omega_i-1})$ with $\sigma_{\omega_i}=\mysem{\chi_{\omega_i-1}}(\sigma_{\omega_i})$. For each OSR transition $\hat{\pi}_i$ has been obtained from $\hat{\pi}_{i-1}$ using \dopasses\ for some sequence $L$ of LVE transformations. Indeed, in order for \ref{eq:mv-big-step} to apply:
\begin{align*}
(\hat{\pi}_{i-1},\hat{\pi}_i)\in{\cal E} ~\wedge~ \exists L: ~&{\tt do\_passes}(\hat{\pi}_{i-1}, L-1)=(\hat{\pi}_i,\mu_{\hat{\pi}_{i-1}\hat{\pi}_i},{\mu'}_{\hat{\pi}_{i}\hat{\pi}_{i-1}}) ~\wedge~\\
&M(\hat{\pi}_{i-1},\hat{\pi}_i)=\mu_{\hat{\pi}_{i-1}\hat{\pi}_i}
\end{align*}

\noindent When the OSR step is performed we thus have:
\begin{equation*}
M(\hat{\pi}_{i-1},\hat{\pi}_i)(l_{\omega_i-1})=\mu_{\hat{\pi}_{i-1}\hat{\pi}_i}(l_{\omega_i-1})=(l_{\omega_i},\chi_{\omega_i-1})
\end{equation*}

\noindent By \ref{th:osr-trans-correctness} function $\mu_{\hat{\pi}_{i-1}\hat{\pi}_i}$ provides a strict OSR mapping between $\hat{\pi}_{i-1}$ and $\hat{\pi}_i$, as all LVE transformations in L are composed into a strict mapping (\ref{co:compose-strict}). Note also that since $\Delta_I$ is being used to map OSR program points between $\hat{\pi}_{i-1}$ and $\hat{\pi}_i$, it follows that $l_{\omega_i}=l_{\omega_i-1}~\forall i\in[1,k]$.
We now prove our claim by induction on $i$.

\paragraph*{Base step.} When $i=1$, we know that no OSR transition has been performed till $l_{\omega_1-1}$ and $\hat{\pi}_0$ has been executing all the time. Then we can write:
\begin{equation*}
(1, \sigma, 1)\trans^*_{\Pi}(1,\sigma_{\omega_1-1},l_{\omega_1-1}) \Longleftrightarrow (\sigma, 1) \trans^*_{\hat{\pi}_0} (\sigma_{\omega_1-1},l_{\omega_1-1})
\end{equation*}

\noindent Trivially, $(\sigma_{\omega_1-1},l_{\omega_1-1})\in\tau_{\hat{\pi}_0\sigma}$. We can thus infer from \ref{de:osr-mapping}:
\begin{align*}
\exists s_j=(\sigma_j,l_j)\in\tau_{\hat{\pi}_1\sigma}~\text{s.t.}~&\mu_{\hat{\pi}_{0}\hat{\pi}_1}(l_{\omega_1-1})=(l_j,\chi)~\wedge \\
&\mysem{\chi}(\sigma_{\omega_1-1}\vert_{\live(\hat{\pi}_{0},\,l_{\omega_1-1})})=\sigma_j\vert_{\live(\hat{\pi}_{1},\,l_j)}
\end{align*}

\noindent From the definition of $\mu_{\hat{\pi}_{0}\hat{\pi}_1}$ it follows that $\chi=\chi_{\omega_1-1}$ and $l_j=l_{\omega_1}=l_{\omega_1-1}$. To prove the claim we need to show that:

\begin{equation*}
\sigma_j\vert_{\live(\hat{\pi}_{1},\,l_{\omega_1})}=\sigma_{\omega_1}\vert_{\live(\hat{\pi}_{1},\,l_{\omega_1})}
\end{equation*}

\noindent which follows directly from \Cref{le:build-comp-corr,le:osr-mapping-comp}.

\paragraph*{Inductive step.} As inductive hypothesis we assume that $\exists (\hat{\sigma}_{k-1},\hat{l}_{k-1})\in\tau_{\hat{\pi}_{k-1}\sigma}$ s.t.:
\begin{equation*}
\hat{l}_{k-1}=l_{\omega_{k-1}} ~\wedge~ \hat{\sigma}_{k-1}\vert_{\live(\hat{\pi}_{k-1},\,\hat{l}_{k-1})}=\sigma_{\omega_{k-1}}\vert_{\live(\hat{\pi}_{k-1},\,\hat{l}_{k-1})}
\end{equation*}

\noindent
Since no OSR is performed between $\tau[\omega_{k-1}]$ and $\tau[\omega_k-1]$ we can write:
\begin{gather*}
(\hat{\sigma}_{k-1}, l_{\omega_{k-1}}) \trans^*_{\hat{\pi}_{k-1}} \cdots \trans^*_{\hat{\pi}_{k-1}} (\tilde{\sigma},l_{\omega_k-1}) \Longleftrightarrow \\
(\sigma_{\omega_{k-1}}, l_{\omega_{k-1}}) \trans^*_{\hat{\pi}_{k-1}} \cdots \trans^*_{\hat{\pi}_{k-1}} (\sigma_{\omega_k-1},l_{\omega_k-1})
\end{gather*}

\noindent in the same number of steps, with $\tilde{\sigma}\vert_{\live(\hat{\pi}_{k-1},\,l_{\omega_k-1})} = \sigma_{\omega_k-1}\vert_{\live(\hat{\pi}_{k-1},\,l_{\omega_k-1})}$ by \ref{le:only-live-count}. Since $(\tilde{\sigma},l_{\omega_k-1})\in\tau_{\hat{\pi}_{k-1}\sigma}$ by the strictness of the OSR mapping $\mu_{\hat{\pi}_{k-1}\hat{\pi}_{k}}$:
\begin{align*}
\exists s_j=(\sigma_j,l_j)\in\tau_{\hat{\pi}_k\sigma}~\text{s.t.}~&\mu_{\hat{\pi}_{k-1}\hat{\pi}_{k}}(l_{\omega_k-1})=(l_j,\chi)~\wedge \\
&\mysem{\chi}(\tilde{\sigma}\vert_{\live(\hat{\pi}_{k-1},\,l_{\omega_k-1})})=\sigma_j\vert_{\live(\hat{\pi}_{k},\,l_j)}
\end{align*}

\noindent From the definition of $\mu_{\hat{\pi}_{k-1}\hat{\pi}_k}$ it follows that $\chi=\chi_{\omega_k-1}$ and $l_j=l_{\omega_k}=l_{\omega_k-1}$. By \Cref{le:build-comp-corr,le:osr-mapping-comp} we thus prove:
\begin{align*}
\sigma_j\vert_{\live(\hat{\pi}_{k},\,l_{\omega_k})} &= \mysem{\chi_{\omega_k-1}}(\tilde{\sigma}\vert_{\live(\hat{\pi}_{k-1},\,l_{\omega_k-1})})\\
&=\mysem{\chi_{\omega_k-1}}(\sigma_{\omega_k-1}\vert_{\live(\hat{\pi}_{k-1},\,l_{\omega_k-1})})\\
&=\sigma_k\vert_{\live(\hat{\pi}_{k},\,l_{\omega_k})})
\end{align*}
\end{myproof}

\begin{theorem}[Multi-Version Program Determinism]
\label{th:mv-prog-determ}
Let $\Pi$ $=({\cal V}, {\cal E}, {\cal M})$ be a multi-version program constructed using \dopasses\ and live-variable equivalent transformations. Then $\Pi$ is deterministic.
\end{theorem}

\begin{myproof}
To prove that $\Pi$ is deterministic, we need to show that, given any initial store $\sigma$ on which $\pi_1\in\Pi$ terminates on some final state $\sigma'=\mysem{\pi_1}(\sigma)$, any execution trace $\tau\in{\cal T}_{\Pi,\sigma}$ terminates with $\sigma'$.

\noindent Let $\omega_1,\ldots,\omega_k$ be the indices of $\tau$ where an OSR transition has just occurred, i.e., for any $i\in[1,k]$, state $\tau[\omega_i]$ is obtained from $\tau[\omega_i-1]$ by applying compensation code $\chi_{\omega_i-1}$ on store $\sigma_{\omega_i-1}$, which yields a store $\sigma_{\omega_i}$. The transition leads from a point $l_{\omega_i-1}$ in version $\pi_{p_{\omega_i-1}}$ to a point $l_{\omega_i}=l_{\omega_i-1}$ in version $\pi_{p_{\omega_{i}}}$ in $\Pi$. 

By \mylemma\ref{le:comp-lemma}, $\forall i\in[1,k]$ there exists a state $(\hat{\sigma}_i,\hat{l}_i)$ in the trace of $\hat{\pi}_i=\pi_{p_{\omega_{i}}}$ starting from the initial store $\sigma$ such that $\hat{l}_i=l_{\omega_i}$ and $\hat{\sigma}_i\vert_{\live(\hat{\pi}_i,\hat{l}_i)}=\sigma_{\omega_i}\vert_{\live(\hat{\pi}_i,\hat{l}_i)}$. Hence, since no OSR is fired after $\omega_k$, by \myequation\ref{eq:mv-big-step} it holds: 
\begin{equation*}
(\hat{\pi}_k,\sigma_{\omega_{k}},l_{\omega_{k}})\trans^*_{\Pi}(\hat{\pi}_k,\sigma',|\hat{\pi}_k|+1) \Longleftrightarrow (\sigma_{\omega_{k}},l_{\omega_{k}})\trans^*_{\hat{\pi}_k}(\sigma',|\hat{\pi}_k|+1)
\end{equation*}

\noindent We can then apply \Cref{le:only-live-count,le:comp-lemma} to write:
\begin{gather*}
(\sigma_{\omega_{k}},l_{\omega_{k}})\trans^*_{\hat{\pi}_k}(\sigma',|\hat{\pi}_k|+1) \Longleftrightarrow \\ 
(\sigma_{\omega_{k}}\vert_{\live(\hat{\pi}_k,l_{\omega_{k}})},l_{\omega_{k}})\trans^*_{\hat{\pi}_k}(\sigma',|\hat{\pi}_k|+1) \Longleftrightarrow \\
(\hat{\sigma}_k\vert_{\live(\hat{\pi}_k,\hat{l}_k)},\hat{l}_k)\trans^*_{\hat{\pi}_k}(\sigma',|\hat{\pi}_k|+1)
\end{gather*}

\noindent As $(\hat{\sigma}_k,\hat{l}_k)\in\tau_{\hat{\pi}_k\sigma}$, by \ref{le:only-live-count} necessarily $\sigma'=\mysem{\hat{\pi}_k}(\sigma)$. 
Given that all programs in $\Pi$ are semantically equivalent, we can conclude that $\mysem{\Pi}(\sigma)=\sigma'=\mysem{\hat{\pi}_k}(\sigma)=\mysem{\pi_1}(\sigma)$.
\end{myproof}


