\section{Proving On-Stack Replacement Sound}

On-stack Replacement is employed in modern adaptive compilation systems to dynamically switch between different versions of a function depending on program's run-time state. Traditionally, code optimizers are responsible for marking the points where such transitions can take place, and generating required meta-data or ad-hoc code to get the program state to a correct resumption point. OSR is usually at the core of large and complex JIT compilers employed by popular production virtual machines. Indeed, the engineering effort to integrate this technique in a language runtime can be daunting, making it rarely accessible to the research community.

\paragraph*{Contributions.} In this thesis we investigate how to provide VM builders with a rich ``menu'' of possible program points where OSR can safely occur, relieving code optimizers from the burden of generating all the required machinery to realign the program state during an OSR transition.

To capture OSR in its full generality, we define a notion of {\em multi-program}, which is a collection of different versions of a program along with support to dynamically transfer execution between them. Execution in a multi-program starts from a designated base version. At any time, an oracle decides whether execution should continue in the current version, or an OSR should divert it to a different version, modeling any conceivable OSR-firing strategy. One of the goals of our work is to characterize sufficient conditions for a multi-program to be {\em deterministic}, yielding the same result regardless of the oracle's decisions. This captures the intuitive idea that any sequence of OSR transitions is {\em correct} if it does not alter the intended semantics of a program.

We distill the essence of OSR to a simple imperative calculus with an operational semantics. Using program bisimulation, we prove that an OSR can correctly divert execution from one program version to the other if they are {\em live-variable bisimilar}, i.e., the live variables they have in common at any corresponding execution states are equal. As prominent examples of how bisimulation can be used to prove this property, we consider classic optimizations that eliminate or move code around, such as dead code elimination, constant propagation, and code hoisting. We show how to construct OSR machinery by devising an algorithm that automatically generates compensation code to reconstruct the values of variables that are live in the OSR target, but not in the source.

\subsection{Language Syntax and Semantics}
\label{ss:osr-language-framework}
Our discussion is based on a minimal imperative programming whose syntax is reported \ifauthorea{below}{in \myfigure\ref{fig:osr-program-syntax}}. In this section we introduce some basic definitions used in our representation of programs, and provide a big-step semantics for the language.

%\subsection{Language Framework}

%In this section we introduce some basic definitions used in our representation of programs and transformations. In particular, we provide a syntax and a big-step semantics for a simple imperative language, and introduce means for reasoning about program properties using computation tree logic operators and for describing program transformations through rewrite rules with side conditions.

%\subsubsection*{Program Syntax and Semantics}

%Our discussion is based on a minimal imperative programming language with instructions and expressions with the syntax \ifauthorea{below}{in \myfigure\ref{fig:osr-program-syntax}}.

\ifdefined\noauthorea
\begin{figure}[t]
\fi
\noindent
\begin{small}
$
\begin{array}{rcl}
\texttt{$Instr$} & ::= & \hphantom{\texttt{| }}\texttt{$Var$ := $Expr$} \\
& & \texttt{| if ( $Expr$ ) goto $Num$} \\
& & \texttt{| goto $Num$} \\
& & \texttt{| skip} \\
& & \texttt{| abort} \\
& & \texttt{| in $Var\cdots Var$} \\ 
& & \texttt{| out $Var\cdots Var$} \\
\texttt{ $Expr$ } & ::= & \texttt{$Num$ | $Var$ | $Expr$ + $Expr$ | $\ldots$ } \\
\texttt{ $Var$ } & ::= & \texttt{X | Y | Z | $\ldots$} \\
\texttt{ $Num$ } & ::= & \texttt{$\ldots$ | -2 | -1 | 0 | 1 | 2 | $\ldots$} \\
\end{array}
$
\end{small}
\ifdefined\noauthorea
\caption{\label{fig:osr-program-syntax}Program Syntax}
\end{figure}
\fi

\begin{definition}[Program]
\label{de:program}
A program is a sequence of instructions the form:
\vspace{-3mm}
\begin{equation*}
\pi=\langle I_1, I_2, \ldots, I_n \rangle\in Prog = \bigcup_{i=2}^{\infty} Instr^{i}
\vspace{-4mm}
\end{equation*}
where: 

\begin{itemize}[itemsep=3pt,parsep=0pt,topsep=3pt]
%\item $Instr$ is defined by the syntax of Figure~\ref{fi:program-syntax}
\item \texttt{$I_i\in Instr$} is the $i$-th instruction of the program, indexed by program point \texttt{$i\in[1,n]$}
\item \texttt{$I_1$ $=$ in $\cdots$} is the initial instruction
\item \texttt{$\forall i\in[2,n-1]:$ $I_i$ $\neq$ in $\cdots$ $\wedge$ $I_i$ $\neq$ out $\cdots$}
\item \texttt{$I_n$ $=$ out $\cdots$} is the final instruction
\end{itemize}
\end{definition}

\noindent Instruction \texttt{in}, which must appear at the beginning of a program, specifies the variables that must be defined prior to entering the program. Similarly, \texttt{out} occurs at the end and specifies the variables that are returned as output. 

By \texttt{e[x]} we indicate that \texttt{x} is a variable of expression \texttt{e}\,$\in Expr$. We also denote by $vars($\texttt{e}$)$ the set of variables that occur in expression \texttt{e}. By $|\pi|=n$ we indicate the number of instructions in $\pi=\langle I_1, I_2, \ldots, I_n \rangle$.

\begin{definition}[Memory Store]
A {\em memory store} is a total function $\sigma:Var\rightarrow \mathbb{Z}\cup\{\bot\}$ that associates integer values to defined variables, and $\bot$ to undefined variables. We denote by $\Sigma$ the set of all possible memory stores. 
\end{definition}

\noindent By $\sigma[\wx\gets v]$ we denote the same function as $\sigma$, except that $\wx$ takes value $v$. Furthermore, for any $A\subseteq Var$, $\sigma\vert_{A}$ denotes $\sigma$ restricted to the variables in $A$, i.e., $\sigma\vert_{A}(\wx)=\sigma(\texttt{x})$ if $\wx\in A$ and $\sigma\vert_{A}(\wx)=\bot$ if $\wx\not\in A$. 

\begin{definition}[Program State]
\label{de:prog-state}
The {\em state} of a program $\pi=\langle I_1, I_2, \ldots, I_n \rangle$ is described by a pair $(\sigma,l)$, where $\sigma$ is a memory store and $l\in [1,n]$ is the program point of the next instruction to be executed. We denote by $State=\Sigma\times \mathbb{N}$ the set of all possible program states.
\end{definition}

\noindent We provide a big-step semantics using transition relation $\trans_{\pi}\:\subseteq State\times State$, which specifies how a single instruction of a program $\pi$ affects its state. Our description relies on relation $\Downarrow\subseteq(\Sigma\times Expr)\times \mathbb{Z}$ to describe how expressions are evaluated in a given memory store.


\begin{definition}[Big-Step Transitions]
\label{de:transitions}
%For any program $\pi$, relation $\Rightarrow_{\pi}\subseteq State\times State$ is defined as follows, with meta-variables $\texttt{x}, \texttt{y}\in Var$, $\texttt{e}\in Expr$, and $\texttt{m}\in Num$:
For any program $\pi$, we define relation $\Rightarrow_{\pi}\:\subseteq State\times State$ as follows, with meta-variables $\texttt{x}, \texttt{y}\in Var$, $\texttt{e}\in Expr$, and $\texttt{m}\in Num$:

\begin{small}

% asgn
\begin{equation}
\label{eq:asgn-sem}
\frac
{I_l=\texttt{x:=e} ~~ \wedge ~~ (\sigma, \texttt{e}) \Downarrow v}
{(\sigma, l)\Rightarrow_{\pi} (\sigma[\wx\gets v], l+1)}
\end{equation}
% if (0)
\vspace{0.5mm}
\begin{equation}
\label{eq:ifz-sem}
\frac
{I_l=\texttt{if (e) goto m} ~~ \wedge ~~ (\sigma, \texttt{e}) \Downarrow 0}
{(\sigma, l)\Rightarrow_{\pi} (\sigma, l+1)}
\end{equation}
% if (!0)
\vspace{0.5mm}
\begin{equation}
\label{eq:ifnz-sem}
\frac
{I_l=\texttt{if (e) goto m} ~~ \wedge ~~ (\sigma, \texttt{e}) \Downarrow v ~~~ \wedge ~~~ v\neq 0}
{(\sigma, l)\Rightarrow_{\pi} (\sigma, \texttt{m})}
\end{equation}
% goto
\vspace{0.5mm}
\begin{equation}
\label{eq:goto-sem}
\frac
{I_l=\texttt{goto m}}
{(\sigma, l)\Rightarrow_{\pi} (\sigma, \texttt{m})}
\end{equation}
% skip
\vspace{0.5mm}
\begin{equation}
\label{eq:skip-sem}
\frac
{I_l=\texttt{skip}}
{(\sigma, l)\Rightarrow_{\pi} (\sigma, l+1)}
\end{equation}
% in
\vspace{0.5mm}
\begin{equation}
\label{eq:in-sem}
\frac
{I_1=\texttt{in x y}~\cdots ~~ \wedge ~~~ \sigma(\texttt{x})\neq\bot ~~~ \wedge ~~~ \sigma(\texttt{y})\neq\bot ~~~ \wedge ~~~ \cdots }
{(\sigma, 1)\Rightarrow_{\pi} (\sigma, 2)}
\end{equation}
% out
%\vspace{0.5mm}
\begin{equation}
\label{eq:out-sem}
\frac
{I_n=\texttt{out x y}~\cdots ~~ \wedge ~~~ \sigma(\texttt{x})\neq\bot ~~~ \wedge ~~~ \sigma(\texttt{y})\neq\bot ~~~ \wedge ~~~ \cdots }
{(\sigma, n)\Rightarrow_{\pi} (\sigma\vert_{\{\texttt{x}, \texttt{y}, \cdots\}}, n+1)}
\end{equation}

\end{small}

\end{definition}

\noindent For a transition to apply, we implicitly assume that $I_l$ is defined, i.e., $l\in[1,n]$. 
\ifx\noauthorea\undefined
Notice that we intentionally do not provide any transition rule for {\tt abort} instructions, providing explicit means to let a program have undefined semantics. This might be useful in supporting unsound speculative optimizations.
\fi

\begin{definition}[Program Semantic Function]
\label{de:program-semantics}
%The semantic function $\mysem{\pi}:\Sigma \rightarrow \Sigma$ of a program $\pi$ is defined as:
We define the semantic function $\mysem{\pi}:\Sigma \rightarrow \Sigma$ of a program $\pi$ as: 
\begin{gather*}
\forall \sigma\in\Sigma: ~~ \mysem{\pi}(\sigma)=\sigma'~~%\vert_{\{\wx\,:\,I_{n}={\tt out}~\cdots~\wx~\cdots\}} ~~ \\ 
\Longleftrightarrow ~~ (\sigma,1) \Rightarrow^{*}_{\pi} (\sigma',|\pi|+1)
\end{gather*}
where $\Rightarrow^{*}_{\pi}$ is the transitive closure of $\Rightarrow_{\pi}$.
\end{definition}

\noindent Note that a program has undefined semantics if its execution on a given store does not reach the final \texttt{out} instruction. This accounts for infinite loops, abort instructions, exceptions, and ill-defined programs or input stores. 

We define the notion of program semantic equivalence as follows:

\begin{definition}[Program Equivalence]
\label{de:semantic-equivalence}
Two programs $\pi_1$ and $\pi_2$ are {\em semantically equivalent} iff $\mysem{\pi_1}=\mysem{\pi_2}$.
\end{definition}

\noindent A notion that will be useful in proving correctness in our framework is that of {\em trace} of a transition system:

\begin{definition}[Traces]
\label{de:exec-trace}
A {\em trace} in a transition system $(S,$ $R\subseteq S^2)$ starting from $s\in S$ is a sequence $\tau=\langle s_0,s_1,\ldots,$ $s_i,\ldots\rangle$ such that $s_0=s$ and $\forall i\ge 0:~s_i\in\tau ~ \wedge ~ s_i~R~s_{i+1}$ $\Longleftrightarrow s_{i+1}\in\tau$. By ${\cal T}_{R,s}$ we denote the system of all traces of $(S,R\subseteq S^2)$ starting from $s$. By $\tau[i]$ we denote the $i$-th state of $\tau$, i.e., $\tau[i]=s_i$. Furthermore, if trace $\tau$ is finite then $|\tau|$ denotes the index of its final state, i.e., $\tau=\langle s_0,s_1,\ldots,s_{|\tau|}\rangle$, otherwise $|\tau|=\infty$. Finally, $dom(\tau)=\{i:s_i\in\tau\}$ denotes the set of indices of states in $\tau$.
\end{definition}

\noindent Notice that, since $\Rightarrow_{\pi}$ is deterministic in our language, then for any initial store $\sigma$, the system of traces ${\cal T}_{\Rightarrow_{\pi},(\sigma,1)}$ of the execution transition system $(Store,\Rightarrow_{\pi})$ contains a single trace, which we denote by $\tau_{\pi\sigma}$.

Finally, we provide a formal definition of control flow graph, which will be useful in defining computation tree logic operators for reasoning on program properties:

\begin{definition}[Control Flow Graph]
\label{de:cfg}
%The {\em control flow graph} G for a program $\pi=\langle I_1, I_2, \ldots, I_n \rangle$ is described by a pair $(V,E)$ where $V = \{ I_1, I_2, \ldots, I_n \}$ and $E = \{(I_i, I_{i+1})\:|\: I_i \neq \textsf{abort} \wedge I_i \neq \textsf{goto m}, \!\textsf{ m}\in Num \}\;\cup\;\{(I_i, I_m)\:|\: I_i = \textsf{goto m} \vee I_i = \textsf{if (e) goto m}, \!\textsf{ m}\in Num, \!\textsf{ e}\in Expr \}$.
%The {\em control flow graph} G for a program $\pi=\langle I_1, I_2, \ldots, I_n \rangle$ is described by a tuple $\langle V, I: V\rightarrow Num, E \subseteq V\times V\rangle$ where:
The {\em control flow graph} $G$ for a program $\pi=\langle I_1, I_2, \ldots, I_n \rangle$ is described by a pair $(V, E \subseteq V\times V)$ where:
%\begin{equation*}
\begin{align*}
V &= \{ I_1, I_2, \ldots, I_n \} \\
%I(v) &= \{ i | v = I_i, v\in V\} \\
E &= \{(I_i, I_{i+1})\:|\: I_i \neq \textsf{abort} \wedge I_i \neq \textsf{goto m}, \!\textsf{ m}\in Num \} \\
&\cup\;\{(I_i, I_m)\:|\: I_i = \textsf{goto m} \vee I_i = \textsf{if (e) goto m}, \!\textsf{ m}\in Num, \!\textsf{ e}\in Expr \}.
\end{align*}
%\end{equation*}
%\noindent We also define an auxiliary function $I: V\rightarrow Num$ that returns the $i$-th index in $\pi$ of an instruction $v\in V$.
\end{definition}

\subsection{On-Stack Replacement Framework}

\subsection{OSR Mapping Generation Algorithm}

\subsection{Multi-Version Programs}

\subsection{LLVM Implementation}

\subsection{Discussion}