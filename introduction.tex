\chapter{Introduction}

Translating programming languages into a form that can {\em efficiently} execute on a target platform is one of the most challenging problems in computer science. Historically, there are two approaches to translation: interpretation and compilation. An interpreter reads the source code of a program, stepping through its expressions to determine which operation is to perform next. A compiler translates a program into a form that is more amenable to execution, analyzing the source code once and generating code that would give the same effects as interpreting the program.

These two approaches show different advantages in terms of execution speed, code footprint, portability, and access to run-time information.
%As programming languages have been evolving over the years,
%Modern interpreters translate source code into an intermediate representation that is easier to work with, and can optionally perform optimizations based on the current workload.

\section{Context and Motivations}

\section{Addressed Problems}

\section{Contributions of the Thesis}

\section{Structure of the Thesis}