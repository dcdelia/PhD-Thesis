\chapter{Introduction}

Translating programming languages into a form that can {\em efficiently} execute on a target platform is a very challenging problem for computer scientists. Historically, there are two approaches to translation: interpretation and compilation. An interpreter reads the source code of a program, stepping through its expressions to determine which operation to perform next. A compiler instead translates a program into a form that is more amenable to execution, analyzing its source code only once and generating code that would give the same effects as interpreting it.

The two approaches have different benefits in terms of execution speed, portability, footprint, and optimization opportunities. Compiled programs typically execute faster, as a compiler can devote an arbitrary amount of time to {\em static} (i.e., prior to run-time) code analysis and optimization. On the other hand, an interpreter can access run-time information such as taken control-flow, input parameter values, and variable types, thus enabling optimizations that static compilation would miss. Indeed, this information may be subject to changes across different runs, or may not be obtainable in general through sole source code inspection.

Additionally, the evolution of programming languages over the years has provided software developers with a plethora of useful features such as dynamic typing and class loading, reflection, and closures that may hinder efficient code generation in a static compiler. In response, industry and academia have significantly invested in {\em adaptive optimization} technology, which consists in observing the run-time behavior of a program in order to drive optimization decisions.

\section{Context and Motivations}

The past two decades have witnessed the widespread adoption of programming languages designed to run on {\em application virtual machines} (VMs). Compared to statically compiled software, these execution environments provide several advantages from a software engineering perspective, including portability, automatic memory and concurrency management, safety, and ease of implementation for {\em dynamic} features of a programming language such as adding new code, extending object definitions, and modifying the type system.

Application virtualization technology has been brought to the mainstream market by the Java programming language and later by the Common Language Runtime for the execution of .NET programs. Virtual machines are nowadays available for every popular language, including JavaScript, MATLAB, Python, R, and Ruby.

Modern virtual machines typically implement a mixed-mode execution environment, in which an interpreter is used for executing portions of a program until it becomes profitable to compile them through {\em just-in-time} (JIT) compilation and continue the execution in the native code. For efficiency reasons, source code is usually translated into an {\em intermediate representation} (IR) - also known as {\em bytecode} - that is easier to analyze and process. Multiple levels of JIT compilation are possible, each with a different trade-off between compilation time and expected code quality.

Adaptive optimization technology is a central element for the performance of runtime systems. JIT compilation indeed does not come for free: a virtual machine should be able to exploit run-time information to tailor optimized code generation to the current workload, so that the expected performance gains can counterbalance the overheads coming from collecting the profile and performing the optimizations.
%a virtual machine should be able to exploit run-time information to tailor optimized code generation to the current workload in order to achieve effective performance improvements. 

Analyzing the run-time behavior of a program is useful also in the context of statically compiled code. {\em Profile-guided optimization} (PGO) techniques adopt a dual-compilation model in which a program is compiled and executed on representative input sets during an initial training stage, and is eventually recompiled using feedback information to generate the final optimized version.

\section{Addressed Problems}

Collecting accurate profiling information with a minimal impact on a running program is a key factor for an effective deployment of adaptive optimization techniques. In principle, developers and VM builders may leverage hardware performance counters provided by modern processors to collect low-level profiling information with no impact on the performance of a running program. However, the difficulty in mapping low-level counter data to high-level constructs such as classes and objects discourages their use for implementing complex analyses in runtime systems.

In the past three decades many sophisticated software-based techniques have been proposed for collecting fine-grained information regarding individual statements, objects, or control-flow paths. These techniques are typically based on program instrumentation, sampling, or a combination of both. For some problems, however, the size of the domain can be particularly large and extant techniques do not scale well or may even run out of space when analyzing real-world programs. In this thesis we investigate how data structure-based techniques from the algorithmic community can be used to devise efficient performance profiling tools. % dire dell'aspetto di focalizzare l'attenzione sul codice da ottimizzare


Another key ingredient for adaptive optimization is the ability of a runtime to divert the execution to the newly generated optimized code {\em while} the original function is executing. In fact, in the presence of long-running methods it is not sustainable for a VM to wait for the original function to complete and let only subsequent invocations run the optimized version. The problem of handling transitions between different compiled versions is formally known as {\em On-Stack Replacement} (OSR). Modern VMs implement OSR techniques to dynamically replace code with (more) optimized code, and also to invalidate aggressive, speculative optimizations and continue in a safe code version when an assumption made during the compilation no longer holds.

Supporting OSR in a runtime system raises a number of fundamental questions. What is the set of program points at which OSR can safely occur, and how is it affected by compiler optimizations? Can we guarantee the soundness of an OSR transition? What is the impact of OSR machinery on running code? In this thesis, we address these questions from both a methodological and a practical perspective.

\section{Contributions of the Thesis}

The first contribution of this thesis is an {\em interprocedural} technique for identifying the most frequently encountered calling contexts\footnote{A {\em calling context} for a function call is defined as the sequence of functions that are concurrently active on the stack when the call is performed.} across function invocations. We show that the traditional approach for constructing the {\em Calling Context Tree} (CCT) might  not be sustainable for real-world applications, as their CCTs might easily consist of tens of millions of node, making them difficult to analyze and also hurting execution time because of poor access locality. We introduce a novel data structure, the {\em Hot Calling Context Tree} (HCCT), in the spectrum of representations for interprocedural control flow. The HCCT is the subtree of the CCT containing only its most visited nodes, which we call {\em hot}, along with their ancestors, and can be constructed independently from the CCT using fast, space-efficient algorithms for mining frequent items in data stream. These algorithms allow us to distinguish between hot and cold context on-the-fly, and we show both theoretically and experimentally that for collected metrics the HCCT achieves a similar precision as the CCT in a space that is several orders of magnitude smaller. We show on prominent benchmarks that our implementation, which ships as a plugin for the gcc compiler, incurs a slowdown competitive with the gprof call-graph profiler while collecting finer-grained profiles.

%In this thesis, we devise two {\em dynamic} (i.e., run-time) analyses for collecting fine-grained profiling information based on efficient and elegant algorithmic techniques. The first analysis works at {\em intra-procedural} level and identifies cyclic paths that are taken in the control-flow graph of a single procedure, thus spanning multiple loop iterations. The second analysis is {\em inter-procedural} as it focuses on identifying the calling contexts of function invocations that are most frequently encountered during the execution of a real-world application, where a {\em calling context} is defined as the sequence of functions concurrently active on the stack when a function call is performed. Both techniques can provide valuable information for program understanding and performance analysis, as they can be used to direct optimizations to portions of the code where most resources are consumed.

%The main contributions of this thesis can 

\section{Structure of the Thesis}


%This thesis tackles problems arising while analyzing the behavior of a running program, within and across the boundaries of a procedure, 

%especially when the target form is directly executable on hardware. Interpreters on the other hand can 
%Modern interpreters translate source code into an intermediate representation that is easier to work with, and can optionally perform optimizations based on the current workload.
