\chapter{Introduction}

Translating programming languages into a form that can {\em efficiently} execute on a target platform is a very challenging problem for computer scientists. Historically, there are two approaches to translation: interpretation and compilation. An interpreter reads the source code of a program, stepping through its expressions to determine which operation to perform next. A compiler instead translates a program into a form that is more amenable to execution, analyzing its source code only once and generating code that would give the same effects as interpreting it.

The two approaches have different benefits in terms of execution speed, portability, footprint, and optimization opportunities. Compiled programs typically execute faster, as a compiler can devote an arbitrary amount of time to {\em static} (i.e., prior to run-time) code analysis and optimization. On the other hand, an interpreter can access run-time information such as taken control-flow, input parameter values, and variable types, thus enabling optimizations that static compilation would miss. Indeed, this information may be subject to changes across different runs, or may not be obtainable in general through sole source code inspection.

Additionally, the evolution of programming languages over the years has provided software developers with a plethora of useful features such as dynamic typing and class loading, reflection, and closures that may hinder efficient code generation in a static compiler. In response, industry and academia have significantly invested in {\em adaptive optimization} technology, which consists in observing the run-time behavior of a program in order to drive optimization decisions.

%This thesis tackles problems arising while analyzing the behavior of a running program, within and across the boundaries of a procedure, 

\section{Context and Motivations}

The past two decades have witnessed the widespread adoption of programming languages designed to run on {\em application virtual machines} (VMs). Compared to statically compiled software, these execution environments provide several advantages from a software engineering perspective, including portability, automatic memory and concurrency management, safety, and ease of implementation for {\em dynamic} features of a programming language such as adding new code, extending object definitions, and modifying the type system.

Application virtualization technology has been brought to the mass market by the Java programming language and later by the Common Language Runtime for the execution of .NET programs. Virtual machines are nowadays available for every popular language, including JavaScript, MATLAB, Python, R, and Ruby.

Modern virtual machines typically implement a mixed-mode execution environment, in which an interpreter is used for executing portions of a program until it becomes profitable to compile them through {\em just-in-time} (JIT) compilation and continue the execution in the native code. For efficiency reasons, source code is usually translated into an {\em intermediate representation} (IR) - also known as {\em bytecode} - that is easier to analyze and process. Multiple levels of JIT compilation are possible, each with a different trade-off between compilation time and expected code quality.

Adaptive optimization technology is a central element for the performance of runtime systems. JIT compilation indeed does not come for free: a virtual machine should be able to exploit run-time information to tailor optimized code generation to the current workload in order to achieve effective performance improvements. 

Analyzing the run-time behavior of a program is useful also in the context of statically compiled code. {\em Profile-guided optimization} (PGO) techniques adopt a dual-compilation model in which a program is compiled and executed on representative input sets during an initial training stage, and is eventually recompiled using feedback information to generate the final optimized version.

%Observing the run-time behavior of a program 

%especially when the target form is directly executable on hardware. Interpreters on the other hand can 

%Modern interpreters translate source code into an intermediate representation that is easier to work with, and can optionally perform optimizations based on the current workload.

\section{Addressed Problems}

Collecting accurate profiling information with a minimal impact on a running program is a key factor for an effective deployment of adaptive optimization techniques. In principle, developers and VM builders may leverage hardware performance counters provided by modern processors to collect low-level profiling information with no impact on a running program. However, the difficulty in mapping low-level counter data to high-level constructs such as classes and objects discourages their use for implementing complex analyses in runtime systems.

\section{Contributions of the Thesis}

\section{Structure of the Thesis}